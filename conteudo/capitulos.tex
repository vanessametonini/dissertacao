%!TeX root=../tese.tex
%("dica" para o editor de texto: este arquivo é parte de um documento maior)
% para saber mais: https://tex.stackexchange.com/q/78101/183146

% Apague as duas linhas abaixo (elas servem apenas para gerar um
% aviso no arquivo PDF quando não há nenhum dado a imprimir) e
% insira aqui o conteúdo dos capítulos do seu trabalho

%\providecommand\aviso[1]{
  \clearpage
  \null
  \vfill
  \begin{hyphenrules}{nohyphenation}
    \centering\bfseries\Large
    #1\par
  \end{hyphenrules}
  \vfill
  \clearpage
}

\providecommand\avisoFolhasDeRosto{
  \aviso{
    {\huge Você precisa editar os arquivos no diretório ``\texttt{conteudo}''!}
    \par\bigskip\bigskip\bigskip\bigskip
    Para gerar a capa e demais páginas preliminares no formato correto,
    modifique os arquivos ``\texttt{conteudo/paginas-preliminares.tex}'' e
    ``\texttt{conteudo/metadados.tex}'', usando como base os arquivos
    correspondentes no diretório ``\texttt{conteudo-exemplo}''.
  }
}

\providecommand\avisoCapitulos{
  \aviso{
    Insira o conteúdo dos capítulos do seu trabalho no arquivo
    ``\texttt{capitulos.tex}'' do diretório ``\texttt{conteudo}''.
  }
}

\providecommand\avisoApendices{
  \aviso{
    Insira o conteúdo dos apêndices do seu trabalho no arquivo
    ``\texttt{apendices.tex}'' do diretório ``\texttt{conteudo}''.
  }
}

\providecommand\avisoAnexos{
  \aviso{
    Insira o conteúdo dos anexos do seu trabalho no arquivo
    ``\texttt{anexos.tex}'' do diretório ``\texttt{conteudo}''.
  }
}

\providecommand\avisoArtigo{
  \aviso{
    Insira o conteúdo do artigo no arquivo ``\texttt{corpo-artigo.tex}''
    do diretório ``\texttt{conteudo}''. Não se esqueça de consultar
    o exemplo no diretório ``\texttt{conteudo-exemplo}'' para a
    definição do título, autoria etc.
  }
}

\providecommand\avisoApresentacao{
  \begin{frame}{Insira o conteúdo!}
  \aviso{
    Insira o conteúdo da apresentação no arquivo ``\texttt{corpo-apresentacao.tex}''
    do diretório ``\texttt{conteudo}''. Não se esqueça de consultar
    o exemplo no diretório ``\texttt{conteudo-exemplo}'' para a
    definição do título, autoria, estrutura etc.
  }
  \end{frame}
}

\providecommand\avisoPoster{
  \aviso{
    Insira o conteúdo do poster no arquivo ``\texttt{corpo-poster.tex}''
    do diretório ``\texttt{conteudo}''. Não se esqueça de consultar
    o exemplo no diretório ``\texttt{conteudo-exemplo}'' para a
    definição do título, autoria, estrutura etc.
  }
}

%\avisoCapitulos
Introdução

%contextualização, contexto da pesquisa motivação, cenário
Apresentação do problema
um pouco do que já foi feito
mas ainda nao existe a verificacao automatica da conformidade
ainda é um problema em aberto
cobrir um gap que hoje


objetivos



Conceitos e fundamentação teórica

(mais livros, porem artigos tb, a survey on...)


tudo que o leitor precisa saber para entender o trabalho
Privacidade
adoção de definição de tal autor/autores
apresentar diferentes definicoes
justificar a escolha

citar artigo que conceito foi criado (por exemplo na GDPR)

LGPD 


Metodologia e proposta

Trabalhos relacionados
banco de dados
aplicações e sistemas
    client > shadow traffic

microsservicos
ferramentas




% Os capítulos de compõem a dissertação/tese, com numeração normal, podem
% ser inseridos diretamente aqui ou "puxados" de outros arquivos.
% Em alguns (raros) casos, pode ser interessante usar \include ao
% invés de \input: https://tex.stackexchange.com/a/32058/183146

%\input{conteudo/01...}
%\par

%\input{conteudo/02...}
%\par
