%!TeX root=../tese.tex
%("dica" para o editor de texto: este arquivo é parte de um documento maior)
% para saber mais: https://tex.stackexchange.com/q/78101/183146

% Apague as duas linhas abaixo (elas servem apenas para gerar um
% aviso no arquivo PDF quando não há nenhum dado a imprimir) e
% insira aqui o conteúdo dos capítulos do seu trabalho

%\input{extras/aviso-conteudo}
%\avisoCapitulos
Introdução

%contextualização, contexto da pesquisa motivação, cenário
Apresentação do problema
um pouco do que já foi feito
mas ainda nao existe a verificacao automatica da conformidade
ainda é um problema em aberto
cobrir um gap que hoje


objetivos



Conceitos e fundamentação teórica

(mais livros, porem artigos tb, a survey on...)


tudo que o leitor precisa saber para entender o trabalho
Privacidade
adoção de definição de tal autor/autores
apresentar diferentes definicoes
justificar a escolha

citar artigo que conceito foi criado (por exemplo na GDPR)

LGPD 


Metodologia e proposta

Trabalhos relacionados
banco de dados
aplicações e sistemas
    client > shadow traffic

microsservicos
ferramentas




% Os capítulos de compõem a dissertação/tese, com numeração normal, podem
% ser inseridos diretamente aqui ou "puxados" de outros arquivos.
% Em alguns (raros) casos, pode ser interessante usar \include ao
% invés de \input: https://tex.stackexchange.com/a/32058/183146

%\input{conteudo/01...}
%\par

%\input{conteudo/02...}
%\par
